% Options for packages loaded elsewhere
\PassOptionsToPackage{unicode}{hyperref}
\PassOptionsToPackage{hyphens}{url}
%
\documentclass[
]{article}
\usepackage{amsmath,amssymb}
\usepackage{iftex}
\ifPDFTeX
  \usepackage[T1]{fontenc}
  \usepackage[utf8]{inputenc}
  \usepackage{textcomp} % provide euro and other symbols
\else % if luatex or xetex
  \usepackage{unicode-math} % this also loads fontspec
  \defaultfontfeatures{Scale=MatchLowercase}
  \defaultfontfeatures[\rmfamily]{Ligatures=TeX,Scale=1}
\fi
\usepackage{lmodern}
\ifPDFTeX\else
  % xetex/luatex font selection
\fi
% Use upquote if available, for straight quotes in verbatim environments
\IfFileExists{upquote.sty}{\usepackage{upquote}}{}
\IfFileExists{microtype.sty}{% use microtype if available
  \usepackage[]{microtype}
  \UseMicrotypeSet[protrusion]{basicmath} % disable protrusion for tt fonts
}{}
\makeatletter
\@ifundefined{KOMAClassName}{% if non-KOMA class
  \IfFileExists{parskip.sty}{%
    \usepackage{parskip}
  }{% else
    \setlength{\parindent}{0pt}
    \setlength{\parskip}{6pt plus 2pt minus 1pt}}
}{% if KOMA class
  \KOMAoptions{parskip=half}}
\makeatother
\usepackage{xcolor}
\usepackage[margin=1in]{geometry}
\usepackage{color}
\usepackage{fancyvrb}
\newcommand{\VerbBar}{|}
\newcommand{\VERB}{\Verb[commandchars=\\\{\}]}
\DefineVerbatimEnvironment{Highlighting}{Verbatim}{commandchars=\\\{\}}
% Add ',fontsize=\small' for more characters per line
\usepackage{framed}
\definecolor{shadecolor}{RGB}{248,248,248}
\newenvironment{Shaded}{\begin{snugshade}}{\end{snugshade}}
\newcommand{\AlertTok}[1]{\textcolor[rgb]{0.94,0.16,0.16}{#1}}
\newcommand{\AnnotationTok}[1]{\textcolor[rgb]{0.56,0.35,0.01}{\textbf{\textit{#1}}}}
\newcommand{\AttributeTok}[1]{\textcolor[rgb]{0.13,0.29,0.53}{#1}}
\newcommand{\BaseNTok}[1]{\textcolor[rgb]{0.00,0.00,0.81}{#1}}
\newcommand{\BuiltInTok}[1]{#1}
\newcommand{\CharTok}[1]{\textcolor[rgb]{0.31,0.60,0.02}{#1}}
\newcommand{\CommentTok}[1]{\textcolor[rgb]{0.56,0.35,0.01}{\textit{#1}}}
\newcommand{\CommentVarTok}[1]{\textcolor[rgb]{0.56,0.35,0.01}{\textbf{\textit{#1}}}}
\newcommand{\ConstantTok}[1]{\textcolor[rgb]{0.56,0.35,0.01}{#1}}
\newcommand{\ControlFlowTok}[1]{\textcolor[rgb]{0.13,0.29,0.53}{\textbf{#1}}}
\newcommand{\DataTypeTok}[1]{\textcolor[rgb]{0.13,0.29,0.53}{#1}}
\newcommand{\DecValTok}[1]{\textcolor[rgb]{0.00,0.00,0.81}{#1}}
\newcommand{\DocumentationTok}[1]{\textcolor[rgb]{0.56,0.35,0.01}{\textbf{\textit{#1}}}}
\newcommand{\ErrorTok}[1]{\textcolor[rgb]{0.64,0.00,0.00}{\textbf{#1}}}
\newcommand{\ExtensionTok}[1]{#1}
\newcommand{\FloatTok}[1]{\textcolor[rgb]{0.00,0.00,0.81}{#1}}
\newcommand{\FunctionTok}[1]{\textcolor[rgb]{0.13,0.29,0.53}{\textbf{#1}}}
\newcommand{\ImportTok}[1]{#1}
\newcommand{\InformationTok}[1]{\textcolor[rgb]{0.56,0.35,0.01}{\textbf{\textit{#1}}}}
\newcommand{\KeywordTok}[1]{\textcolor[rgb]{0.13,0.29,0.53}{\textbf{#1}}}
\newcommand{\NormalTok}[1]{#1}
\newcommand{\OperatorTok}[1]{\textcolor[rgb]{0.81,0.36,0.00}{\textbf{#1}}}
\newcommand{\OtherTok}[1]{\textcolor[rgb]{0.56,0.35,0.01}{#1}}
\newcommand{\PreprocessorTok}[1]{\textcolor[rgb]{0.56,0.35,0.01}{\textit{#1}}}
\newcommand{\RegionMarkerTok}[1]{#1}
\newcommand{\SpecialCharTok}[1]{\textcolor[rgb]{0.81,0.36,0.00}{\textbf{#1}}}
\newcommand{\SpecialStringTok}[1]{\textcolor[rgb]{0.31,0.60,0.02}{#1}}
\newcommand{\StringTok}[1]{\textcolor[rgb]{0.31,0.60,0.02}{#1}}
\newcommand{\VariableTok}[1]{\textcolor[rgb]{0.00,0.00,0.00}{#1}}
\newcommand{\VerbatimStringTok}[1]{\textcolor[rgb]{0.31,0.60,0.02}{#1}}
\newcommand{\WarningTok}[1]{\textcolor[rgb]{0.56,0.35,0.01}{\textbf{\textit{#1}}}}
\usepackage{graphicx}
\makeatletter
\def\maxwidth{\ifdim\Gin@nat@width>\linewidth\linewidth\else\Gin@nat@width\fi}
\def\maxheight{\ifdim\Gin@nat@height>\textheight\textheight\else\Gin@nat@height\fi}
\makeatother
% Scale images if necessary, so that they will not overflow the page
% margins by default, and it is still possible to overwrite the defaults
% using explicit options in \includegraphics[width, height, ...]{}
\setkeys{Gin}{width=\maxwidth,height=\maxheight,keepaspectratio}
% Set default figure placement to htbp
\makeatletter
\def\fps@figure{htbp}
\makeatother
\setlength{\emergencystretch}{3em} % prevent overfull lines
\providecommand{\tightlist}{%
  \setlength{\itemsep}{0pt}\setlength{\parskip}{0pt}}
\setcounter{secnumdepth}{-\maxdimen} % remove section numbering
\ifLuaTeX
  \usepackage{selnolig}  % disable illegal ligatures
\fi
\IfFileExists{bookmark.sty}{\usepackage{bookmark}}{\usepackage{hyperref}}
\IfFileExists{xurl.sty}{\usepackage{xurl}}{} % add URL line breaks if available
\urlstyle{same}
\hypersetup{
  hidelinks,
  pdfcreator={LaTeX via pandoc}}

\author{}
\date{\vspace{-2.5em}}

\begin{document}

\hypertarget{intro-to-dat-science---hw-2}{%
\section{Intro to Dat Science - HW
2}\label{intro-to-dat-science---hw-2}}

\hypertarget{copyright-jeffrey-stanton-jeffrey-saltz-and-jasmina-tacheva}{%
\subparagraph{Copyright Jeffrey Stanton, Jeffrey Saltz, and Jasmina
Tacheva}\label{copyright-jeffrey-stanton-jeffrey-saltz-and-jasmina-tacheva}}

\begin{Shaded}
\begin{Highlighting}[]
\CommentTok{\# Enter your name here: Mark Cappiello}
\end{Highlighting}
\end{Shaded}

\hypertarget{attribution-statement-choose-only-one-and-delete-the-rest}{%
\subsubsection{Attribution statement: (choose only one and delete the
rest)}\label{attribution-statement-choose-only-one-and-delete-the-rest}}

\begin{Shaded}
\begin{Highlighting}[]
\CommentTok{\# 1. I did this homework by myself, with help from the book and the professor.}
\end{Highlighting}
\end{Shaded}

\hypertarget{reminders-of-things-to-practice-from-last-week}{%
\subsubsection{Reminders of things to practice from last
week:}\label{reminders-of-things-to-practice-from-last-week}}

Assignment arrow \textless- The combine command c( ) Descriptive
statistics mean( ) sum( ) max( ) Arithmetic operators + - * / Boolean
operators \textgreater{} \textless{} \textgreater= \textless= == !=

\textbf{This Week:} Explore the \textbf{quakes} dataset (which is
included in R). Copy the \textbf{quakes} dataset into a new dataframe
(call it \textbf{myQuakes}), so that if you need to start over, you can
do so easily (by copying quakes into myQuakes again). Summarize the
variables in \textbf{myQuakes}. Also explore the structure of the
dataframe

\begin{Shaded}
\begin{Highlighting}[]
\NormalTok{myQuakes }\OtherTok{\textless{}{-}}\NormalTok{ quakes}

\FunctionTok{head}\NormalTok{(myQuakes)}
\end{Highlighting}
\end{Shaded}

\begin{verbatim}
##      lat   long depth mag stations
## 1 -20.42 181.62   562 4.8       41
## 2 -20.62 181.03   650 4.2       15
## 3 -26.00 184.10    42 5.4       43
## 4 -17.97 181.66   626 4.1       19
## 5 -20.42 181.96   649 4.0       11
## 6 -19.68 184.31   195 4.0       12
\end{verbatim}

\textbf{Step 1:} Explore the earthquake magnitude variable called
\textbf{mag}

\begin{enumerate}
\def\labelenumi{\Alph{enumi}.}
\tightlist
\item
  What is the average magnitude? Use mean() or summary():
\end{enumerate}

\begin{Shaded}
\begin{Highlighting}[]
\NormalTok{avgMag }\OtherTok{\textless{}{-}} \FunctionTok{mean}\NormalTok{(myQuakes}\SpecialCharTok{$}\NormalTok{mag)}
\NormalTok{avgMag}
\end{Highlighting}
\end{Shaded}

\begin{verbatim}
## [1] 4.6204
\end{verbatim}

\begin{enumerate}
\def\labelenumi{\Alph{enumi}.}
\setcounter{enumi}{1}
\tightlist
\item
  What is the magnitude of the largest earthquake? Use max() or
  summary() and save the result in a variable called \textbf{maxQuake}:
\end{enumerate}

\begin{Shaded}
\begin{Highlighting}[]
\NormalTok{maxQuake }\OtherTok{\textless{}{-}} \FunctionTok{max}\NormalTok{(myQuakes}\SpecialCharTok{$}\NormalTok{mag)}
\NormalTok{maxQuake}
\end{Highlighting}
\end{Shaded}

\begin{verbatim}
## [1] 6.4
\end{verbatim}

\begin{enumerate}
\def\labelenumi{\Alph{enumi}.}
\setcounter{enumi}{2}
\tightlist
\item
  What is the magnitude of the smallest earthquake? Use min() or
  summary() and save the result in a variable called \textbf{minQuake}:
\end{enumerate}

\begin{Shaded}
\begin{Highlighting}[]
\NormalTok{minQuakes }\OtherTok{\textless{}{-}} \FunctionTok{min}\NormalTok{(myQuakes}\SpecialCharTok{$}\NormalTok{mag)}
\NormalTok{minQuakes}
\end{Highlighting}
\end{Shaded}

\begin{verbatim}
## [1] 4
\end{verbatim}

\begin{enumerate}
\def\labelenumi{\Alph{enumi}.}
\setcounter{enumi}{3}
\tightlist
\item
  Output the \textbf{third row} of the dataframe
\end{enumerate}

\begin{Shaded}
\begin{Highlighting}[]
\NormalTok{myQuakes[}\DecValTok{3}\NormalTok{,]}
\end{Highlighting}
\end{Shaded}

\begin{verbatim}
##   lat  long depth mag stations
## 3 -26 184.1    42 5.4       43
\end{verbatim}

E. Create a new dataframe, with only the rows where the
\textbf{magnitude is greater than 4}. How many rows are in that
dataframe (use code, do not count by looking at the output)

\begin{Shaded}
\begin{Highlighting}[]
\NormalTok{magGreater4 }\OtherTok{\textless{}{-}}\NormalTok{ myQuakes[myQuakes}\SpecialCharTok{$}\NormalTok{mag }\SpecialCharTok{\textgreater{}} \DecValTok{4}\NormalTok{, ]}
\FunctionTok{min}\NormalTok{(magGreater4}\SpecialCharTok{$}\NormalTok{mag)}
\end{Highlighting}
\end{Shaded}

\begin{verbatim}
## [1] 4.1
\end{verbatim}

\begin{Shaded}
\begin{Highlighting}[]
\NormalTok{numRows }\OtherTok{\textless{}{-}} \FunctionTok{nrow}\NormalTok{(magGreater4)}
\NormalTok{numRows}
\end{Highlighting}
\end{Shaded}

\begin{verbatim}
## [1] 954
\end{verbatim}

\begin{enumerate}
\def\labelenumi{\Alph{enumi}.}
\setcounter{enumi}{5}
\tightlist
\item
  Create a \textbf{sorted dataframe} based on magnitude and store it in
  \textbf{quakeSorted1}. Do the sort two different ways, once with
  arrange() and then with order()
\end{enumerate}

\begin{Shaded}
\begin{Highlighting}[]
\CommentTok{\#{-}{-}{-} directions say to create two new dataframe called quakeSorted1}
\CommentTok{\#{-}{-}{-} this seems like a typo but I\textquotesingle{}m going to do what it says}

\FunctionTok{library}\NormalTok{(dplyr)}
\end{Highlighting}
\end{Shaded}

\begin{verbatim}
## 
## Attaching package: 'dplyr'
\end{verbatim}

\begin{verbatim}
## The following objects are masked from 'package:stats':
## 
##     filter, lag
\end{verbatim}

\begin{verbatim}
## The following objects are masked from 'package:base':
## 
##     intersect, setdiff, setequal, union
\end{verbatim}

\begin{Shaded}
\begin{Highlighting}[]
\NormalTok{quakeSorted1 }\OtherTok{\textless{}{-}} \FunctionTok{arrange}\NormalTok{(myQuakes, mag)}

\NormalTok{quakeSorted1 }\OtherTok{\textless{}{-}}\NormalTok{ myQuakes[}\FunctionTok{order}\NormalTok{(myQuakes}\SpecialCharTok{$}\NormalTok{mag), ]}
\end{Highlighting}
\end{Shaded}

\begin{enumerate}
\def\labelenumi{\Alph{enumi}.}
\setcounter{enumi}{6}
\item
  What are the latitude and longitude of the quake reported by the
  largest number of stations?
\item
  What are the latitude and longitude of the quake reported by the
  smallest number of stations?
\end{enumerate}

\textbf{Step 3:} Using conditional if statements

\begin{enumerate}
\def\labelenumi{\Roman{enumi}.}
\tightlist
\item
  Test if \textbf{maxQuake} is greater than 7 (output ``yes'' or ``no'')
  \textbf{Hint:} Try modifying the following code in R:
\end{enumerate}

\begin{Shaded}
\begin{Highlighting}[]
\ControlFlowTok{if}\NormalTok{  (}\DecValTok{100} \SpecialCharTok{\textless{}} \DecValTok{150}\NormalTok{) }\StringTok{"100 is less than 150"} \ControlFlowTok{else} \StringTok{"100 is greater than 150"}
\end{Highlighting}
\end{Shaded}

\begin{verbatim}
## [1] "100 is less than 150"
\end{verbatim}

\begin{enumerate}
\def\labelenumi{\Alph{enumi}.}
\setcounter{enumi}{9}
\tightlist
\item
  Following the same logic, test if \textbf{minQuake} is less than 3
  (output ``yes'' or ``no''):
\end{enumerate}

\end{document}
